\section{Conclusion}
\label{sec:conclusion}


In this report a selection of data analysis techniques have been applied to a dataset to identify patterns and predict behaviour. From this unusual behaviour has been identified and investigated to find employees that may be connected with the disappearance.  
The data provided was used to find the coordinates of the establishments where employees used their credit cards. A complete car assignment was also found by finding the exact identification of the truck drivers. \\

An anomaly was discovered in the spending patterns of the employees. One employee was found to spend considerably more money than any other employee with a similar job.

\noindent Analysis of the times when employees moved showed five employees moving at night when all other employees made no journeys. One of these employees was returning to the Gastech office, no other employees appear to work at night so this was identified as an anomaly. Clustering showed the four other employees were staying outside executives’ houses overnight and meeting with a fifth employee and five unknown locations. 

An employee was identified as travelling a greater distance than other employees. This was confirmed as an anomaly when it was discovered that this employee also uniquely visited four clusters.

Using the probabilities from Section \ref{sec:predicting}, region 3, which is the area that GAStech is located in, is shown to be the area that people frequent most often. This is understandable, as the majority of employees would have to go to the complex every day for work. From the area with GAStech, the next most likely location that they will travel to is area 8, which is full of restaurants and cafes. This is most likely because the employees would go for lunch, or to dinner straight after work. \\
